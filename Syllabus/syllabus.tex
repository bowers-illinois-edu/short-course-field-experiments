\documentclass[letterpaper]{inzane_syllabus} % a4paper for A4

\usepackage{booktabs, colortbl, xcolor}
\usepackage{tabularx}
\usepackage{enumitem}
\usepackage{ltablex}
\usepackage{multirow}

\setlist{nolistsep}

\usepackage{lscape}
\newcolumntype{r}{>{\hsize=0.9\hsize}X}
\newcolumntype{w}{>{\hsize=0.6\hsize}X}
\newcolumntype{m}{>{\hsize=.9\hsize}X}

\renewcommand{\familydefault}{\sfdefault}
\renewcommand{\arraystretch}{2.0}
%----------------------------------------------------------------------------------------
%	 PERSONAL INFORMATION
%----------------------------------------------------------------------------------------

%\profilepic{fish.jpg} % Profile picture, if the height of the picture is less than that of the cirle, it will have a flat bottom.
\profilepic{}

% To remove any of the following, you need to comment/delete the lines in the .cls file (c. line 186). Commenting/deleting the lines below will produce an error.

%To add different lines, you will need to create the new command, e.g. \profPhone, in the .cls file c. line 76, and command to create the line in the side bar in the .cls file c. line 186

\classname{Field Experiments \\ \footnotesize{Version of \today} \\ (DRAFT)}
\classnum{EUI Short Course \\ Summer 2021}

%%%%%%%%%%%%%%% PROF INFO
\profname{Jake Bowers}
\officehours{Office Hours: \href{http://calendly.com/jakebowers}{http://calendly.com/jakebowers}}
\office{Zoom}
\site{http://jakebowers.org}
\email{jwbowers@illinois.edu}

%%%%%%%%%%%%%%% COURSE  INFO
\prereq{}
\classdays{Monday to Friday}
\classhours{15:30--18:00 Florence CET}
\classloc{We will meet on Zoom}

%%%%%%%%%%%%%%% LAB INFO
%\labdays{Wed \& Fri}
%\labhours{2-5p}
%\labloc{Lab Space}
%
%%%%%%%%%%%%%%%% TA INFO
%\taAname{Alice}
%\taAofficehours{Office Hrs: Tues \& Thurs 10-11a}
%\taAoffice{MCZ 104}
%% \taAemail{}
%\taBname{James}
%\taBofficehours{Office Hrs: Tues \& Thurs 3-4p}
%\taBoffice{MCZ 104}
% \taBemail{}

% \about{Fish make up the largest group of vertebrates on the planet, easily outnumbering mammals, marsupials, birds, and reptiles combined. Not only are they abundant, but they've diversified into an extraordinary array of sizes, shapes, lifestyles, and habitats. You can find them in the coldest, deepest parts of the ocean, and in the hottest freshwater ponds in the desert. This course will explore fish diversity and their biology. }


%---------------------------------------------------------------------------------------
%	 FAQs
%----------------------------------------------------------------------------------------
%to add more questions or remove this section, go to the .cls file and start with lines comment
%lines 226-250. Also comment out this section as well as line 152(ish), the command \makeSide

\qOne{How to connect for office hours?}
\aOne{I am trying to use appointments on \href{http://calendly.com/jakebowers}{http://calendly.com/jakebowers} for office hours.}

\qTwo{This syllabus will be revised}
\aTwo{Please check in periodically to see if the syllabus has been revised.}
%
% \qThree{What is your favorite fish?}
% \aThree{A lumpsucker. They are incredibly, adorably weird-looking.}
%
% \qFour{What's the difference between plural `fish' and `fishes'?}
% \aFour{`Fish' is the plural form when talking about two or more fish of the same species. `Fishes' is the plural when talking about two or more different species.}

%----------------------------------------------------------------------------------------

\begin{document}

%----------------------------------------------------------------------------------------
%	 DESCRIPTION
%----------------------------------------------------------------------------------------

\makeprofile % Print the sidebar

%----------------------------------------------------------------------------------------
%	 OVERVIEW
%----------------------------------------------------------------------------------------
\section{Overview}



\subsection{Learning Objectives}

This course aims to help graduate students in the social sciences to create
experimental designs and to analyze the data arising from experiments. Because
we focus on field experiments, and because each field experiments is unique in
its context and challenges, we emphasize the statistical theory that would guide 

%use \begin{outline} or \begin{outline}[enumerate] to create a list with subitems.
\begin{itemize}
    \item Understand why one might manipulate an intervention so that some people or groups of people receive it and others do not. And understand the ethics of experimentation: harm to people, benefits to society.
    \item Understand the concepts of counterfactual causal inference and relate them to the task of learning about social science theories.
    \item Understand how randomization relates to counterfactual causal inference.
    \item Understand how randomization and experimental design can justify hypothesis testing choices and how hypothesis tests play a role in detecting causal effects.
    \item Undersand the concepts of power and false positive rates of hypothesis tests
    \item Understand how estimation can justify estimation choices and how estimation plays a role in detecting causal effects
    \item Understand the concepts of bias and precision of estimation.
    \item Understand the relationship between confidence intervals and hypothesis tests (and thus, how to use confidence intervals in power calculations in experimental design).
    \item Understand the Complier Average Causal Effect or Local Average Treatment Effect and how to estimate it.
    \item Undersand the problems that arise from missing data on outcomes or missing data on covariates.
    \item Understand the uses of covariates in increasing the power of hypothesis tests and/or the precision of estimation in randomized experiments.
    \item Understand the use of pre-analysis plans.
\end{itemize}


%\newpage % Start a new page

%\makeSide % Print the FAQ sidebar; To get rid of, simply comment out and uncomment \makeFullPage

%\makeFullPage

%\section{Goals and Expectations}

\newpage % Start a new page

%\makeSide % Print the FAQ sidebar; To get rid of, simply comment out and uncomment \makeFullPage

\makeFullPage

%%%%%%%%%%%%%%%%%%%%%%%%%%%%%%%%%%%%%%%%%%%%%%%%%%%%%%%%%%%%%%%%%%%%%%%%%%%%%
%                COURSE SCHEDULE
%%%%%%%%%%%%%%%%%%%%%%%%%%%%%%%%%%%%%%%%%%%%%%%%%%%%%%%%%%%%%%%%%%%%%%%%%%%%%
%\newpage
%\makeFullPage
\pdfbookmark[0]{Schedule}{schedule} \section{Class Schedule}
\SetDate[07/06/2021]

\begin{center}
    %\begin{tabularx}{\textwidth}{>{\raggedright}p{2.5cm}  @{\hskip 0.5cm} >{\raggedright}p{7.5cm}>{\raggedright\arraybackslash}p{9.5cm}} %change the width of the comments by changing these cm measurements. Add/substract columns by adding/deleting p{} sections.
    %\begin{tabular}{>{\raggedright}p{2.5cm}  @{\hskip 0.5cm} >{\raggedright}p{7.5cm}>{\raggedright\arraybackslash}p{9.5cm}}        \arrayrulecolor{myCOLOR}\toprule
    \begin{longtable}{>{\raggedright}p{2.5cm}  @{\hskip 0.5cm} >{\raggedright}p{6cm}>{\raggedright\arraybackslash}p{11cm}}        \arrayrulecolor{myCOLOR}\toprule
        %%%%%%%%%%%%%%%%%%%%%%%%%%%%%%%%%%%%%%%%%%% MODULE 1
       % \multicolumn{3}{l}{\textbf{\textcolor{myCOLOR}{\large MODULE 1: Fiction, Scenarios, Prototypes to help us think about politics, society and economics.}}} \\ %\hline
        % Week & Topic & Readings \\ \hline
        %%Alternatively, instead of Week #, you can do Class date for meeting
        %\SetDate[25/08/2020] \DayOfWeek, \themonth~\the\day  & Introduction to the course and each other &

        \syldate{\today}  & Introduction to the course and each other 
         Topics
        \begin{itemize}
            \item What is a randomized control trial?
            \item How do experiments fit into the research design process?
            \item What do we mean by ``causal inference''?
            \item What is the role of randomization in causal inference?
        \end{itemize}


                          & 

       
        Read:
        \begin{itemize}

            \item \cite{gerber2012field}[Chapter 1 and 2]

            \item  \cite{bowersVoorsIchino2021book}[Module 2 and 3]

            \item  \cite{rosenbaum2017}[Chapter 1 and 2]
        \end{itemize}

              \\ \midrule

        %\DayOfWeek, \themonth~\the\day & Introduction to the course and each other & \\ %\midrule
        \AdvanceDate[1] \syldate{\today}  & Hypothesis Testing &  Read 
         \begin{itemize}
            \item \cite{rosenbaum2017}[Chapter 3]
            \item \cite{bowersVoorsIchino2021book}[Module on Hypothesis testing]
            \item \href{https://egap.org/resource/10-things-to-know-about-hypothesis-testing/}{EGAP Hypothesis Testing Guide}
            \item \href{https://egap.org/resource/10-things-to-know-about-multiple-comparisons/}{EGAP Multiple Comparisons Guide}
        \end{itemize}
\\ 
                                          \midrule

       % \multicolumn{3}{l}{\textbf{\textcolor{myCOLOR}{\large MODULE 2: Politics, Parties, Leaders}}} \\ %\hline

        \AdvanceDate[1] \syldate{\today} & Power &  \cite{bowersVoorsIchino2021book}[Module on Statistical Power] \\ \midrule

        \AdvanceDate[1] \syldate{\today} & Estimation & Estimators, Estimates, ATE, LATE/CACE \\ \midrule

        \AdvanceDate[1]\syldate{\today} & Missing Data & Read  
        \\ \midrule

        \AdvanceDate[1]\syldate{\today} & Workshop & Workshop \\ 
        
        \bottomrule
    \end{longtable}
\end{center}

%----------------------------------------------------------------------------------------

%----------------------------------------------------------------------------------------
%	 READING MATERIAL
%----------------------------------------------------------------------------------------
\vspace{0.5cm} %I make liberal use of the \vspace{} command to partition and place sections just how I want them. Alter as you see fit.
\pdfbookmark[0]{Bibliography}{bibliography} \section{Readings}
%\defbibnote{note}{Here are some additional references for further reading}

%\printbibliography[title={Sources},prenote=note]

\printbibliography[heading=none]

\end{document}




\documentclass[letterpaper]{inzane_syllabus} % a4paper for A4

\usepackage{booktabs, colortbl, xcolor}
\usepackage{tabularx}
\usepackage{enumitem}
\usepackage{ltablex}
\usepackage{multirow}

\setlist{nolistsep}

\usepackage{lscape}
\newcolumntype{r}{>{\hsize=0.9\hsize}X}
\newcolumntype{w}{>{\hsize=0.6\hsize}X}
\newcolumntype{m}{>{\hsize=.9\hsize}X}

\renewcommand{\familydefault}{\sfdefault}
\renewcommand{\arraystretch}{2.0}
%----------------------------------------------------------------------------------------
%	 PERSONAL INFORMATION
%----------------------------------------------------------------------------------------

%\profilepic{fish.jpg} % Profile picture, if the height of the picture is less than that of the cirle, it will have a flat bottom.
\profilepic{}

% To remove any of the following, you need to comment/delete the lines in the .cls file (c. line 186). Commenting/deleting the lines below will produce an error.

%To add different lines, you will need to create the new command, e.g. \profPhone, in the .cls file c. line 76, and command to create the line in the side bar in the .cls file c. line 186

\classname{Field Experiments \\ \footnotesize{Version of \today} \\ (DRAFT)}
\classnum{EUI Short Course \\ Summer 2021}

%%%%%%%%%%%%%%% PROF INFO
\profname{Jake Bowers}
\officehours{Office Hours: \href{http://calendly.com/jakebowers}{http://calendly.com/jakebowers}}
\office{Zoom}
\site{http://jakebowers.org}
\email{jwbowers@illinois.edu}

%%%%%%%%%%%%%%% COURSE  INFO
\prereq{}
\classdays{Monday to Friday}
\classhours{15:30--18:00 Florence CET}
\classloc{We will meet on Zoom}

%%%%%%%%%%%%%%% LAB INFO
%\labdays{Wed \& Fri}
%\labhours{2-5p}
%\labloc{Lab Space}
%
%%%%%%%%%%%%%%%% TA INFO
%\taAname{Alice}
%\taAofficehours{Office Hrs: Tues \& Thurs 10-11a}
%\taAoffice{MCZ 104}
%% \taAemail{}
%\taBname{James}
%\taBofficehours{Office Hrs: Tues \& Thurs 3-4p}
%\taBoffice{MCZ 104}
% \taBemail{}

% \about{Fish make up the largest group of vertebrates on the planet, easily outnumbering mammals, marsupials, birds, and reptiles combined. Not only are they abundant, but they've diversified into an extraordinary array of sizes, shapes, lifestyles, and habitats. You can find them in the coldest, deepest parts of the ocean, and in the hottest freshwater ponds in the desert. This course will explore fish diversity and their biology. }


%---------------------------------------------------------------------------------------
%	 FAQs
%----------------------------------------------------------------------------------------
%to add more questions or remove this section, go to the .cls file and start with lines comment
%lines 226-250. Also comment out this section as well as line 152(ish), the command \makeSide

\qOne{How to connect for office hours?}
\aOne{I am trying to use appointments on \href{http://calendly.com/jakebowers}{http://calendly.com/jakebowers} for office hours.}

% \qTwo{What is a fish?}
% \aTwo{No clue. When someone says `fish', we have a picture of a general fish of a general shape in our minds, but the truth is that `fish' doesn't have scientific meaning. Here's a funny video about that: \href{https://youtu.be/uhwcEvMJz1Y}{Youtube (hyperlink)}. }
%
% \qThree{What is your favorite fish?}
% \aThree{A lumpsucker. They are incredibly, adorably weird-looking.}
%
% \qFour{What's the difference between plural `fish' and `fishes'?}
% \aFour{`Fish' is the plural form when talking about two or more fish of the same species. `Fishes' is the plural when talking about two or more different species.}

%----------------------------------------------------------------------------------------

\begin{document}

%----------------------------------------------------------------------------------------
%	 DESCRIPTION
%----------------------------------------------------------------------------------------

\makeprofile % Print the sidebar

%----------------------------------------------------------------------------------------
%	 OVERVIEW
%----------------------------------------------------------------------------------------
\section{Overview}

How can imagining the future help us understand the present? How does
considering the future help us think critically about politics today?  In this
course we will read social science and political philosophy together with
science fiction in an attempt to enhance the political, social and economic
imagination of the social sciences. The future hopes and imaginings of past
political thinkers do not include either enough detail or enough information
about our rapidly changing technological, social, political, and economic
landscape to provide us with enough practice to confidently confront the
future as it happens to us. Science fiction allows us a much more
detailed view of life in alternative futures, and the writers that we choose
to read here tend to think seriously and logically about how current cutting-edge technology might have social and political ramifications --- however,
science fiction authors are also mostly focusing on plot, characters and narrative and thus may
skim over core concepts that ought to organize our thinking about politics and
society. Thus, we read both together in order to practice a kind of
theoretically informed futurism (which is not the same as prediction or
forecasting, but is more like practicing confronting the unexpected).

I think of a college seminar as a kind of guided self-study group. The students
arrive because they are interested in a topic broadly. My job is to help focus
the reading and discussion: I have read more of these pieces than you have, and
I have given some extra thought to the question about how participants in a
self-study group might get the most from their participation in such a group. To
this end, this course offers multiple opportunities for reflection and
engagement with the material: if you don't get it the first time, you'll have
multiple opportunities to work to get it.

\subsection{Learning Objectives}

%use \begin{outline} or \begin{outline}[enumerate] to create a list with subitems.
\begin{itemize}
    \item Practice imagining alternative political, social, and economic futures. It is too easy to imagine future technology and too hard to imagine future political parties and constitutions and systems of taxation and corporate governance. This class aims to help you practice this kind of social, political, economic imagining.
    \item Engage with science fiction novels that attempt to do this kind of task --- to imagine different politics, societies, and economies. (Using the discussion questions, inclass discussion, and your own reading.)
    \item Engage with political philosophers who provide some frameworks for thinking about politics, society and economics that you can use to inform and structure your imagining.  (Using the discussion questions, inclass discussion, and your own reading.)
    \item Engage with non-fiction and commentary about possible future politics, economics, and society.  (Using the discussion questions, inclass discussion, and your own reading.)
    \item Practice writing your own scenarios or prototypes for future politics, economics and society. (Using the three assigned projects.)
\end{itemize}


\newpage % Start a new page

%\makeSide % Print the FAQ sidebar; To get rid of, simply comment out and uncomment \makeFullPage

\makeFullPage

\section{Goals and Expectations}

You will read \emph{both} a science fiction novel \emph{and} a short piece of
political philosophy or other bit of non-fiction almost every week (scheduled
to occur over a weekend for discussion on Tuesdays and use in small group
discussions on Thursdays). Sometimes I will assign a movie or perhaps a short
story instead of a novel.  I hope that the fiction can help us talk more
clearly about and/or understand more deeply the political theory or non-fiction
social scientific work.

I am imagining that we will all talk together on Tuesdays for the whole
classtime about the readings assigned for discussion that day. Then on
Thursdays, we will start the time with any questions you may have and then you
will break into small groups to brainstorm and prototype some scenarios or design fiction,
and then each group will report back to the overall class for discussion. The
emphasis on Thursday is on creating new ideas based on the readings that we
talked about on Tuesdays.

These ideas about how the course goes could change as the course develops.

In order for you and your colleagues to get the most out of this class, I have
designed the following requirements:

\subsection{Attendance}

I require regular attendance. I will consider an absence excused if you have an appropriate note from a dean, doctor, or lawyer.  See \href{http://admin.illinois.edu/policy/code/article1_part5_1-501.html}{the official University policy on absences}. You may have three unexcused absences this term after the first week of classes. Other absences will count as zeros in your in-class involvement grade. Please let me know if you are having technical problems. I hope that the University has resources available to help you connect via both video and audio into our Zoom meetings if you are having problems.

I will expect you to have your video on during the class except during breaks
unless you talk with me about the reasons why you cannot have your video on.

\subsection{Discussion questions}

By 5pm the evening before each Tuesday class, each person will have submitted a
discussion question on the class Moodle. The point of this assignment is to (1)
ensure that the quality of our in class discussions is high, (2) provide some
impetus for you to make time to do the reading, (3) let me know what you are
thinking about the material and (4) give chances for shy folks to get credit
for quality class participation --- by writing thoughtful discussion questions.
These questions will serve as the starting point for group discussion, so they
should not be simple factual or yes/no questions. I will grade them as
unsatisfactory(C-73), satisfactory(B-83), excellent(A-93) taking into account
\emph{engagement with the material}, \emph{understanding of the material}, and
\emph{writing skills}. An excellent question shows deep reading \emph{and}
creative thinking: it does not have to be long. An unsatisfactory question
shows little engagement with the reading, little comprehension of it, and/or
poor writing. Poor writing alone is enough for a low grade. If you are worried
about your writing, you might try checking your question using an online
resource such as \href{http://www.hemingwayapp.com/}{Hemingway} or perhaps make
us of some of the University's writing resources.

You will not be able to turn in these questions late, since I will
use the time between the deadline and class time to read your questions.

In calculating the grade based on the reading questions, I'll drop
your lowest four scores. This means that you can either skip the
assignment four times with no penalty or you can turn in
ill-considered or poorly-written questions four times with no penalty.

\subsection{Involvement/Participation}
Quality class participation does not mean ``talking a lot.''  It
includes turning in assignments on time; attending classes; arriving
on time; thinking and caring about the material and expressing your
thoughts respectfully and succinctly in class.

The best in-class participation that I have seen has come from people
who have done the reading carefully and then listen closely to their
classmates and respond thoughtfully (if possibly critically or
supportively). This class is not a place to make speeches. Nor is it a
place to sit in silence. You get credit for daring to guess or giving
unexpected answers. Although I will call for volunteers to answer the
questions I pose, I will probably call on you if you consistently
don't raise your hand or if you keep avoiding my gaze.

Of course, a discussion class via Zoom is a new thing for us all. So we will
all be learning how to best participate in discussions together this term. So,
my policy here is not set in stone.

\subsection{Short Papers: Scenarios or Design Fictions}

Although most of the course involves close engagement with reading and topics defined here in the syllabus, it is important you have a chance to engage with the material on your own as well as a place to practice using your imagination on your own. I will ask that you write three 2 page singlespaced papers throughout the term. I envision that these papers would be \href{https://en.wikipedia.org/wiki/Design_fiction}{design fictions} or speculative prototypes of some kinds --- in which you write a short piece or produce a short podcast or video from the perspective of a future in such a way that we learn about that future and its upsides and downsides.  For example you could write as if you lived in some future --- perhaps writing a letter to a friend, or a party platform (or a letter or report about an experience with a political party), or something else (a comic, a short video, a podcast see for example the \href{http://roseveleth.com/}{Flash Forward} podcast for a series of short fictions that start many episodes.). For another set of examples of such work, see the \href{https://www.nytimes.com/spotlight/future-oped}{Op-ed from the future series of the New York Times}.

I am currently imagining that each paper would be about 2 pages singlespaced of
the fiction itself (you as the future author) and then 1 page singlespaced
where you explain how one or more of the political theory readings or
nonfiction readings from the class (or that you bring in from outside the
class) informed your choices in the design fiction.

If you want to submit a video or podcast or comic (like an advertisement from
the future, a video tutorial for how to live with/enjoy/cope with a future
object or situation) that is also fine. Pleast talk with me about it first so
that we can figure out how to evaluate it.

\subsection{Presentation}

The last week or so of the course will be devoted to presentations of one of
your three scenarios or design fictions to the rest of the class.  You can choose which one
you want to present. I currently imagine that each person will have roughly 5
minutes to present using slides or some other presentation tool, but we could
change this if we wanted to increase the time devoted to presentations.


%----------------------------------------------------------------------------------------
%	 GRADING SCHEME
%----------------------------------------------------------------------------------------
%   \vspace{0.5cm}
%   \subsection{Grading Scheme}
%   
%   %below is the \twentyshort environment - a list with only two inputs. However, there is a \twenty environment, which creates a list with four inputs. You can find/alter details of that table in the .cls file c. lines 320.
%   
%   I'm currently planning to calculate your grade this way: 15\% for
%   Attendance/Class Participation (recorded on Moodle), 15\% Discussion Questions
%   (graded on Moodle), 65\% scenarios or design fictions (each paper about 65/3 \%), 5\%
%   Presentation
%   
%   I do not curve. If all of you perform excellently, then I will say so
%   to the computer system of the University. That said, I am a hard
%   grader with very high standards: I have never given all As, or even
%   mostly As. I hope I can assign all A's this term.
%   
%   All written work in this class will assume familiarity with the principles of
%   good writing in \autocite{howardsbecker1986a}. If you do not know why one should avoid the
%   passive voice, ask me in class or in office hours and I will post relevant
%   chapters from Becker on the topic. You may want to paste bits of your text into
%   \href{Hemingway}{http://www.hemingwayapp.com/} as well in order to check your
%   writing clarity.
%   
%   There will be only one extra credit opportunity for the class. Successful participation in the Political Science Subject Pool is worth 2 percentage points (to the overall course grade). An announcement with more details will be made in class in September.
%   
%   \subsection{Conduct}
%   
%   I expect you to observe the
%   \href{http://admin.illinois.edu/policy/code/article1_part4_1-402.html}{University
%   of Illinois Campus Code of Conduct} when writing your papers (and in general).
%   So, for example, I expect that you will use the words of others without proper attribution. Violations will lead to a failing grade for that assignment. If you have any questions about what counts as plagiarism, ask the professor.
%   
%   \subsection{Diversity and Inclusivity}
%   
%   I expect that all members of this class contribute to a respectful, welcoming and inclusive environment for every other member of the class.
%   
%   \subsection{Students with disabilities}
%   
%   Contact me as soon as possible (and definitely within the first 2 weeks of class) to request any accommodations needed.
%   
%   \subsection{Emergency Response Instructions}
%   
%   \href{http://police.illinois.edu/emergency-preparedness/run-hide-fight/}{University policy is Run, Hide, Fight.}
%   
%   \subsection{Computing and Writing}
%   
%   The discussion questions on the Moodle will
%   mostly just be typed directly into the Moodle text editor although I will
%   recommend that you type them first in a plain text editor on your device first
%   --- so that an internet glitch doesn't cause you to lose work..
%   
%   Your design fictions can be turned in as pdf documents or shared with me as
%   Google Docs (or using Box documents or Dropbox Paper) --- or if they are
%   multimedia, in some other appropraite format. I do not want to download Word
%   documents. Please do not turn in Word documents.

%----------------------------------------------------------------------------------------
%	 EXTRAS
%----------------------------------------------------------------------------------------

%%%%%%%%%%%%%%%%%%%%%%%%%%%%%%%%%%%%%%%%%%%%%%%%%%%%%%%%%%%%%%%%%%%%%%%%%%%%%
%                SECOND PAGE
%%%%%%%%%%%%%%%%%%%%%%%%%%%%%%%%%%%%%%%%%%%%%%%%%%%%%%%%%%%%%%%%%%%%%%%%%%%%%

\newpage % Start a new page

%\makeSide % Print the FAQ sidebar; To get rid of, simply comment out and uncomment \makeFullPage

\makeFullPage

%%%%%%%%%%%%%%%%%%%%%%%%%%%%%%%%%%%%%%%%%%%%%%%%%%%%%%%%%%%%%%%%%%%%%%%%%%%%%
%                COURSE SCHEDULE
%%%%%%%%%%%%%%%%%%%%%%%%%%%%%%%%%%%%%%%%%%%%%%%%%%%%%%%%%%%%%%%%%%%%%%%%%%%%%
%\newpage
%\makeFullPage
\pdfbookmark[0]{Schedule}{schedule} \section{Class Schedule}
\SetDate[07/06/2021]

\begin{center}
    %\begin{tabularx}{\textwidth}{>{\raggedright}p{2.5cm}  @{\hskip 0.5cm} >{\raggedright}p{7.5cm}>{\raggedright\arraybackslash}p{9.5cm}} %change the width of the comments by changing these cm measurements. Add/substract columns by adding/deleting p{} sections.
    %\begin{tabular}{>{\raggedright}p{2.5cm}  @{\hskip 0.5cm} >{\raggedright}p{7.5cm}>{\raggedright\arraybackslash}p{9.5cm}}        \arrayrulecolor{myCOLOR}\toprule
    \begin{longtable}{>{\raggedright}p{2.5cm}  @{\hskip 0.5cm} >{\raggedright}p{6cm}>{\raggedright\arraybackslash}p{11cm}}        \arrayrulecolor{myCOLOR}\toprule
        %%%%%%%%%%%%%%%%%%%%%%%%%%%%%%%%%%%%%%%%%%% MODULE 1
       % \multicolumn{3}{l}{\textbf{\textcolor{myCOLOR}{\large MODULE 1: Fiction, Scenarios, Prototypes to help us think about politics, society and economics.}}} \\ %\hline
        % Week & Topic & Readings \\ \hline
        %%Alternatively, instead of Week #, you can do Class date for meeting
        %\SetDate[25/08/2020] \DayOfWeek, \themonth~\the\day  & Introduction to the course and each other &

        \syldate{\today}  & Introduction to the course and each other & 

        Topics
        \begin{itemize}
            \item What is a randomized control trial?
            \item  How do experiments fit into the research design process?
            \item  What do we mean by ``causal inference''?
            \item   What is the role of randomization?
        \end{itemize}

        Read:
        \begin{itemize}

            \item \cite{gerber2012field}[Chapter 1 and 2]

            \item  \cite{bowersVoorsIchino2021book}[Module 2 and 3]

            \item  \cite{rosenbaum2017}[Chapter 1 and 2]
        \end{itemize}

              \\ \midrule

        % The freedom to speculate about consequences
        %\url{http://www.slate.com/articles/technology/future_tense/2011/01/the_purpose_of_science_fiction.single.html}

        % The need for utopias
        %\url{http://www.antipope.org/charlie/blog-static/2010/12/utopia.html}

        %\DayOfWeek, \themonth~\the\day & Introduction to the course and each other & \\ %\midrule
        \AdvanceDate[1] \syldate{\today}  & Hypothesis Testing &  Read  \\
                                          && Watch \\ 
                                          \midrule

       % \multicolumn{3}{l}{\textbf{\textcolor{myCOLOR}{\large MODULE 2: Politics, Parties, Leaders}}} \\ %\hline

        \AdvanceDate[1] \syldate{\today} & Power &  \fullcite{bowersVoorsIchino2021book}[Module on Statistical Power] \\ \midrule

        \AdvanceDate[1] \syldate{\today} & Estimation & Estimators, Estimates, ATE, LATE/CACE \\ \midrule

        \AdvanceDate[1]\syldate{\today} & Missing Data & Read  
        \\ \midrule

        \AdvanceDate[1]\syldate{\today} & Workshop & Workshop \\ 
        
        \bottomrule
    \end{longtable}
\end{center}


%----------------------------------------------------------------------------------------


%----------------------------------------------------------------------------------------
%	 READING MATERIAL
%----------------------------------------------------------------------------------------
\vspace{0.5cm} %I make liberal use of the \vspace{} command to partition and place sections just how I want them. Alter as you see fit.
\pdfbookmark[0]{Bibliography}{bibliography} \section{Some Readings}
%\defbibnote{note}{Here are some additional references for further reading}

%\printbibliography[title={Sources},prenote=note]

\printbibliography[heading=none]

\end{document}




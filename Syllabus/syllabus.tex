\documentclass[letterpaper]{inzane_syllabus} % a4paper for A4

\usepackage{booktabs, colortbl, xcolor}
\usepackage{tabularx}
\usepackage{enumitem}
\usepackage{ltablex}
\usepackage{multirow}

\setlist{nolistsep}

%\usepackage{lscape}
%\newcolumntype{r}{>{\hsize=0.9\hsize}X}
%\newcolumntype{w}{>{\hsize=0.6\hsize}X}
%\newcolumntype{m}{>{\hsize=0.9\hsize}X}

\renewcommand{\familydefault}{\sfdefault}
\renewcommand{\arraystretch}{2.0}
%----------------------------------------------------------------------------------------
%	 PERSONAL INFORMATION
%----------------------------------------------------------------------------------------

%\profilepic{fish.jpg} % Profile picture, if the height of the picture is less than that of the cirle, it will have a flat bottom.
\profilepic{}

% To remove any of the following, you need to comment/delete the lines in the .cls file (c. line 186). Commenting/deleting the lines below will produce an error.

%To add different lines, you will need to create the new command, e.g. \profPhone, in the .cls file c. line 76, and command to create the line in the side bar in the .cls file c. line 186

\classname{Field Experiments\\ \footnotesize{Version of \today} \\ (DRAFT)}
\classnum{EUI Short Course \\ Summer 2021}

%%%%%%%%%%%%%%% PROF INFO
\profname{Jake Bowers}
\officehours{Office Hours: \href{http://calendly.com/jakebowers}{http://calendly.com/jakebowers}}
\office{Zoom}
\site{http://jakebowers.org}
\email{jwbowers@illinois.edu}

%%%%%%%%%%%%%%% COURSE  INFO
\prereq{}
\classdays{Monday to Friday}
\classhours{15:30--18:00 Florence CET}
\classloc{We will meet on Zoom}

%%%%%%%%%%%%%%% LAB INFO
%\labdays{Wed \& Fri}
%\labhours{2-5p}
%\labloc{Lab Space}
%
%%%%%%%%%%%%%%%% TA INFO
%\taAname{Alice}
%\taAofficehours{Office Hrs: Tues \& Thurs 10-11a}
%\taAoffice{MCZ 104}
%% \taAemail{}
%\taBname{James}
%\taBofficehours{Office Hrs: Tues \& Thurs 3-4p}
%\taBoffice{MCZ 104}
% \taBemail{}

% \about{Fish make up the largest group of vertebrates on the planet, easily outnumbering mammals, marsupials, birds, and reptiles combined. Not only are they abundant, but they've diversified into an extraordinary array of sizes, shapes, lifestyles, and habitats. You can find them in the coldest, deepest parts of the ocean, and in the hottest freshwater ponds in the desert. This course will explore fish diversity and their biology. }


%---------------------------------------------------------------------------------------
%	 FAQs
%----------------------------------------------------------------------------------------
%to add more questions or remove this section, go to the .cls file and start with lines comment
%lines 226-250. Also comment out this section as well as line 152(ish), the command \makeSide

\qOne{How to connect for office hours?}
\aOne{I am trying to use appointments on \href{http://calendly.com/jakebowers}{http://calendly.com/jakebowers} for office hours.}

\qTwo{This syllabus will be revised}
\aTwo{Please check in periodically to see if the syllabus has been revised.}
%
% \qThree{What is your favorite fish?}
% \aThree{A lumpsucker. They are incredibly, adorably weird-looking.}
%
% \qFour{What's the difference between plural `fish' and `fishes'?}
% \aFour{`Fish' is the plural form when talking about two or more fish of the same species. `Fishes' is the plural when talking about two or more different species.}

%----------------------------------------------------------------------------------------

\begin{document}

%----------------------------------------------------------------------------------------
%	 DESCRIPTION
%----------------------------------------------------------------------------------------

\makeprofile% Print the sidebar

%----------------------------------------------------------------------------------------
%	 OVERVIEW
%----------------------------------------------------------------------------------------
\section{Overview}

This short course aims to help social scientists to create randomized
experimental designs and to analyze the data arising from such studies.
Because we focus on randomized field experiments, and because each field
experiment is unique in its context and challenges, the course emphasizes
statistical theory that would empower scholars to make creative choices in
design and analysis. The course relies on simulation using the R statistical
computing environment to develop intuitions about randomization, testing,
estimation, and statistical power.



\subsection{Learning Objectives}
%use \begin{outline} or \begin{outline}[enumerate] to create a list with subitems.

By the end of the class, I hope that students will be able to answer the following questions:

\begin{itemize}
    \item Why would a social scientist manipulate an intervention so that some
        people or groups of people receive it and others do not? When might we
        \emph{not} want to experiment?  What are common approaches to
        preventing harm to experimental subjects?
    \item What does "X causes Y" mean in the counterfactual causal framework?
        What is a "potential outcome"? How might learning about counterfactual
        causal relationships help us understand our theories better? 
    \item Why might we use randomization to learn about counterfactual causal
        relationships?
    \item How might randomization and experimental design justify hypothesis
        testing choices? How might hypothesis tests play a role in helping us
        detecting causal effects?
    \item What is statistical power? What is the false positive rate of a
        hypothesis test? What is the family wise error rate of multiple
        hypothesis tests? Why should we care about these kinds of
        characteristics of tests?
    \item How might randomization and experimental design justify estimation
        choices? How might estimation play a role in learning about causal
        effects?
    \item What is bias in an estimator? What is precision of an estimator? Why
        should we care about these kinds of characteristics of estimators?
    \item What is the relationship the relationship between confidence
        intervals and hypothesis tests?     
    \item What is the Complier Average Causal Effect or Local Average Treatment Effect? When might we want to learn about this causal effect? How might we use estimation and/or testing to learn about it?
    \item What are some useful tactics for dealing with the problems that arise from missing data on outcomes versus missing data on covariates?
    \item How might one use of covariates to increasing the power of hypothesis tests and/or the precision of estimation in randomized experiments either in the design phase or outcome analysis phase of the research?
    \item What is a pre-analysis plans? Why do researchers and organizations use them?
\end{itemize}

I also hope that student will have sketched a research design for a randomized experiment following the \href{https://egap.github.io/theory_and_practice_of_field_experiments/the-research-design-process.html}{EGAP Research Design Form}.

\subsection{Class Plans}

I have found that methods classes are most fun when a student has a project to
which the material might be directly applied.

I have also found that they are most fun when the class sessions are filled
with questions and discussion.

Because we only have a week, I will probably begin each session with about 30
mins of presentation, allowing for an hour of discussion from the class. And
then another hour of work by students on on their own research designs ---
where I am available for discussion and to help confront the many code errors
that will arise.


%\newpage % Start a new page

%\makeSide % Print the FAQ sidebar; To get rid of, simply comment out and uncomment \makeFullPage

%\makeFullPage

%\section{Goals and Expectations}

\newpage % Start a new page

%\makeSide % Print the FAQ sidebar; To get rid of, simply comment out and uncomment \makeFullPage

\makeFullPage

%%%%%%%%%%%%%%%%%%%%%%%%%%%%%%%%%%%%%%%%%%%%%%%%%%%%%%%%%%%%%%%%%%%%%%%%%%%%%
%                COURSE SCHEDULE
%%%%%%%%%%%%%%%%%%%%%%%%%%%%%%%%%%%%%%%%%%%%%%%%%%%%%%%%%%%%%%%%%%%%%%%%%%%%%
%\newpage
%\makeFullPage
\pdfbookmark[0]{Schedule}{schedule} \section{Class Schedule}
\SetDate[07/06/2021]

\begin{center}
    %\begin{tabularx}{\textwidth}{>{\raggedright}p{2.5cm}  @{\hskip 0.5cm} >{\raggedright}p{7.5cm}>{\raggedright\arraybackslash}p{9.5cm}} %change the width of the comments by changing these cm measurements. Add/substract columns by adding/deleting p{} sections.
    %\begin{tabular}{>{\raggedright}p{2.5cm}  @{\hskip 0.5cm} >{\raggedright}p{7.5cm}>{\raggedright\arraybackslash}p{9.5cm}}        \arrayrulecolor{myCOLOR}\toprule
    \begin{longtable}{>{\raggedright}p{1.5cm} @{\hskip 0.25cm} >{\raggedright} p{10cm} >{\raggedright\arraybackslash} p{8cm}} \arrayrulecolor{myCOLOR} \toprule
        %%%%%%%%%%%%%%%%%%%%%%%%%%%%%%%%%%%%%%%%%%% MODULE 1
        % \multicolumn{3}{l}{\textbf{\textcolor{myCOLOR}{\large MODULE 1: Fiction, Scenarios, Prototypes to help us think about politics, society and economics.}}} \\ %\hline
        % Week & Topic & Readings \\ \hline
        %%Alternatively, instead of Week #, you can do Class date for meeting
        %\SetDate[25/08/2020] \DayOfWeek, \themonth~\the\day  & Introduction to the course and each other &
        \syldate{\today}  & Introductions, Causal Inference, Randomization 

        \begin{itemize}
            \item Introduction to the course and each other
            \item Why would a social scientist manipulate an intervention so
                that some people or groups of people receive it and others do
                not? When might we \emph{not} want to experiment?  What are
                common approaches to preventing harm to experimental subjects?
            \item What does "X causes Y" mean in the counterfactual causal
                framework? What is a "potential outcome"? How might learning
                about counterfactual causal relationships help us understand
                our theories better? 
            \item Why might we use randomization to learn about counterfactual
                causal relationships?
            \item How to use R to randomly assign units in different types of randomized designs: simple, complete, blocked, clustered.
        \end{itemize}
                          & 
                          Read:
                          \begin{itemize}
                              \item \cite{gerber2012field}[Chapter 1 and 2]
                              \item \cite{bowersVoorsIchino2021book}[Modules on Causal Inference and Randomization] (See also the links to EGAP Methods Guides)
                              \item \cite{rosenbaum2017}[Chapter 1 and 2]
                          \end{itemize}

                          \\ \midrule

                          %\DayOfWeek, \themonth~\the\day & Introduction to the course and each other & \\ %\midrule
                          \AdvanceDate[1] \syldate{\today}  & Statistical Inference about Causal Effects: Hypothesis Testing and Estimation

     \begin{itemize}
         \item Why test a hypothesis to learn about the causal effect of an
             experimental intervention?
         \item How might randomization and experimental design justify hypothesis testing choices?  How might hypothesis tests play a role in helping us detecting causal effects?
         \item What is the false positive rate of a hypothesis test?  What is the family wise error rate of multiple hypothesis tests? Why should we care about these characteristics of tests?
         \item Why estimate an average causal effect?
         \item How might randomization and experimental design justify estimation choices? How might estimation play a role in learning about causal effects?
         \item What is bias in an estimator? What is precision of an estimator? Why should we care about these kinds of characteristics of estimators?
         \item What is the relationship the relationship between confidence intervals and hypothesis tests?     
         \item (Maybe) How might one use covariates to increase the power of hypothesis tests and/or the precision of estimation in randomized experiments either in the design phase or outcome analysis phase of the research?
         \item How to use R to test hypotheses about counterfactual causal effects in randomized experiments.
         \item How to use R to estimate average counterfactual causal effects in randomized experiments.
     \end{itemize}

                 &  
                 Read: 
                 \begin{itemize}
                     \item \cite{rosenbaum2017}[Chapter 3]
                     \item \cite{bowersVoorsIchino2021book}[Module on Hypothesis testing] (Including links to EGAP Guides)
                                                                                          \item \cite{gerber2012field}[Chapter 3 \& 4]
                    \item \cite{bowersVoorsIchino2021book}[Module on Estimation]  (Including links to EGAP Guides)
                    \item (Maybe) \cite{bowers2020causality}
                 \end{itemize}
                 \\ 
                                                            \midrule

                          \AdvanceDate[1] \syldate{\today} & Power
                          \begin{itemize}
                              \item What is statistical power? Why should we care about statistical power? 
                              \item How to do power analysis using R and DeclareDesign?
                          \end{itemize}

                                                           &  Read:
                                                           \begin{itemize}
                                                               \item \cite{bowersVoorsIchino2021book}[Module on Statistical Power] (see also the EGAP Methods Guides linked therein)
                                                               \item Something from the \href{https://declaredesign.org}{DeclareDesign website} or \href{https://book.declaredesign.org}{DeclareDesign book}
                                                           \end{itemize}
                                                           \\ \midrule

                          \AdvanceDate[1]\syldate{\today} & Non-Compliance \& Missing Data 

                          \begin{itemize}
                              \item What is the Complier Average Causal Effect or Local Average Treatment Effect? When might we want to learn about this causal effect? How might we use estimation and/or testing to learn about it?
                              \item What are some useful tactics for dealing with the problems that arise from missing data on outcomes versus missing data on covariates?
                              \item How to estimate and test hypotheses about the CACE/LATE using R?
                          \end{itemize}

  & 
  Read:
  \begin{itemize}
      \item \cite{gerber2012field}[Chapter 5 and 6]
      \item \cite{bowersVoorsIchino2021book}[Module on Threats]
  \end{itemize}

  \\ \midrule

                          \AdvanceDate[1]\syldate{\today} & Pre-analysis plans \& Workshop 

                          \begin{itemize}
                              \item What is a pre-analysis plan? Why do researchers and organizations use them?
                              \item (Depending on class size) Students present research designs for class discussion.
                          \end{itemize}


                                                          & Read:
                                                          \begin{itemize}
                                                              \item \href{https://egap.org/resource/10-things-to-know-about-pre-analysis-plans/}{EGAP guide to Pre-Analysis Plans}
                                                          \end{itemize}
                                                          \\ 

                                                          \bottomrule
                      \end{longtable}
                  \end{center}

                  %----------------------------------------------------------------------------------------

                  %----------------------------------------------------------------------------------------
                  %	 READING MATERIAL
                  %----------------------------------------------------------------------------------------
                  \vspace{0.5cm} %I make liberal use of the \vspace{} command to partition and place sections just how I want them. Alter as you see fit.
                  \pdfbookmark[0]{Bibliography}{bibliography} \section{Readings}
                  %\defbibnote{note}{Here are some additional references for further reading}

                  %\printbibliography[title={Sources},prenote=note]

                  \printbibliography[heading=none]

                  \end{document}



